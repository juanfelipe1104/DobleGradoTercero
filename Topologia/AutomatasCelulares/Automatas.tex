\documentclass[12pt]{article}
\usepackage[a4paper,margin=2.5cm]{geometry}
\usepackage{amsmath, amsfonts, amssymb, amsthm}
\usepackage{physics}
\usepackage{mathtools}

\title{Continuidad en autómatas celulares y su interpretación topológica}
\author{}
\date{}

\begin{document}

\maketitle

\section*{1. Configuraciones y estructura topológica}

Un \textbf{autómata celular} es un sistema dinámico discreto definido sobre una red de celdas, donde cada celda puede adoptar un valor de un conjunto finito de estados.

\subsection*{Espacio de configuraciones}

Sea:
\[
A = \{a_1, a_2, \dots, a_k\} \quad \text{(conjunto finito de estados)},
\]
y sea $\mathbb{Z}^d$ la red $d$-dimensional de índices de celdas.

Una \textbf{configuración} es una función:
\[
x : \mathbb{Z}^d \longrightarrow A, \qquad i \longmapsto x(i),
\]
que asigna a cada celda su estado.

El conjunto de todas las configuraciones posibles se denota:
\[
X = A^{\mathbb{Z}^d}.
\]

Cada elemento $x \in X$ representa un ``estado global'' del autómata.

\subsection*{Topología sobre $X$}

Se dota a $A$ de la \textbf{topología discreta}.  
La topología natural sobre $X$ es la \textbf{topología producto discreta}, la más débil que hace continuas las proyecciones:
\[
\pi_i : X \to A, \quad \pi_i(x) = x(i).
\]

\subsection*{Base de abiertos}

Los abiertos básicos se denominan \textbf{cilindros} y tienen la forma:
\[
C(i_1, \dots, i_k; a_1, \dots, a_k)
= \{\, x \in X : x(i_j) = a_j,\; j=1,\dots,k \,\}.
\]

En esta topología:
\begin{itemize}
  \item Dos configuraciones son \textbf{cercanas} si coinciden en un gran número de posiciones alrededor de un punto central.
  \item Cuanto mayor sea la región de coincidencia, más ``cercanas'' son.
\end{itemize}

Esta noción de cercanía refleja la idea física de que un cambio local en una celda no debería alterar inmediatamente la configuración global.

\bigskip

\section*{2. Regla local y evolución global}

Sea $N \subset \mathbb{Z}^d$ un conjunto finito (vecindario).  
La \textbf{regla local} es una función:
\[
f : A^N \longrightarrow A,
\]
que determina el nuevo estado de una celda en función de los valores de sus vecinas.

La \textbf{evolución global} del autómata se define como:
\[
F : X \longrightarrow X, \qquad (F(x))(i) = f\big(x|_{i+N}\big),
\]
donde $x|_{i+N}$ denota la restricción de $x$ al bloque $i+N$.

\bigskip

\section*{3. Continuidad de $F$ en la topología producto (canónica)}

\subsection*{Proposición}
La función global $F : A^{\mathbb{Z}^d} \to A^{\mathbb{Z}^d}$ es continua en la topología producto discreta.

\subsection*{Demostración}

Sea $U$ un cilindro en la imagen:
\[
U = \{\, y \in X : y(i_j) = b_j,\; j=1,\dots,k \,\}.
\]

Entonces:
\[
F^{-1}(U)
= \{\, x \in X : (F(x))(i_j) = b_j,\; j=1,\dots,k \,\}.
\]

Por la definición de $F$:
\[
(F(x))(i_j) = f(x|_{i_j + N}),
\]
de modo que:
\[
F^{-1}(U)
= \bigcap_{j=1}^{k} \{\, x \in X : f(x|_{i_j + N}) = b_j \,\}.
\]

Cada conjunto del tipo $\{ x : f(x|_{i_j + N}) = b_j \}$ restringe solo un número \textbf{finito} de coordenadas.  
Como $A$ tiene la topología discreta, fijar finitas coordenadas define un conjunto abierto (una unión de cilindros).

Por tanto, cada uno de los conjuntos anteriores es abierto, y su intersección finita también lo es.

\[
\boxed{F^{-1}(U) \text{ es abierto } \Rightarrow F \text{ es continua.}}
\]

\subsection*{Interpretación}

La \textbf{localidad finita} de la regla $f$ garantiza que la imagen de una configuración solo depende de un número finito de valores.  
Esto hace que la evolución global $F$ sea continua en la topología producto: configuraciones ``cercanas'' (iguales en una gran región finita) evolucionan hacia configuraciones igualmente cercanas.

\[
\text{Pequeños cambios locales} \;\Rightarrow\; \text{pequeños cambios en la evolución.}
\]

\bigskip

\section*{4. Ejemplo numérico: Regla 90 (caso unidimensional)}

Consideremos $A=\{0,1\}$, $\mathbb{Z}$ como red de celdas y $N=\{-1,0,1\}$ como vecindario.  
La regla local está dada por:
\[
f(a_{i-1}, a_i, a_{i+1}) = a_{i-1} + a_{i+1} \pmod 2.
\]

Entonces:
\[
(F(x))(i) = f(x(i-1), x(i), x(i+1)).
\]

Sea el cilindro:
\[
U = \{\, y \in X : y(0)=1,\, y(1)=0 \,\}.
\]

Su preimagen bajo $F$ es:
\[
F^{-1}(U)
= \{\, x : f(x(-1),x(0),x(1))=1, \, f(x(0),x(1),x(2))=0 \,\}.
\]

Sustituyendo $f$:
\[
\begin{cases}
x(-1) + x(1) \equiv 1 \pmod 2,\\[3pt]
x(0) + x(2) \equiv 0 \pmod 2.
\end{cases}
\quad \Rightarrow \quad
\begin{cases}
x(-1)\neq x(1),\\
x(0)=x(2).
\end{cases}
\]

Estas condiciones afectan solo a las coordenadas $\{-1,0,1,2\}$, un conjunto finito.  
Cada combinación de valores que las satisface define un cilindro abierto, y la unión de todos ellos es abierta.

\[
\boxed{F^{-1}(U) \text{ es abierto } \Rightarrow F \text{ es continua.}}
\]

\bigskip

\section*{5. Contraejemplo: no continuidad con la topología de semirrectas derechas}

\subsection*{Definición}

La \textbf{topología de semirrectas derechas} sobre $X=A^{\mathbb{Z}}$ se define por la base:
\[
U_N(x) = \{\, y \in X : y(i)=x(i) \text{ para todo } i \ge N \,\}.
\]

En esta topología, dos configuraciones son ``cercanas'' si coinciden completamente a partir de cierto punto hacia la derecha.

\subsection*{Ejemplo}

Consideremos nuevamente la regla local de la Regla 90:
\[
F(x)(i) = x(i-1) + x(i+1) \pmod 2.
\]

Sea el abierto en la imagen:
\[
V = \{\, y : y(0)=0 \,\}.
\]
Entonces:
\[
F^{-1}(V) = \{\, x : x(-1)=x(1) \,\}.
\]

\subsection*{Prueba de no continuidad}

Tomemos la configuración nula $x_0$, definida por $x_0(i)=0$ para todo $i$.  
Entonces $x_0 \in F^{-1}(V)$.

Supongamos que $F$ fuera continua.  
Entonces existiría algún $N$ tal que $U_N(x_0) \subseteq F^{-1}(V)$.

Sin embargo, para cualquier $N$, definimos $y$ como:
\[
y(i) = 
\begin{cases}
1, & i=-1,\\
0, & \text{en otro caso.}
\end{cases}
\]
Entonces $y \in U_N(x_0)$ (porque coincide con $x_0$ para todo $i \ge N$),  
pero:
\[
(F(y))(0) = y(-1) + y(1) = 1 + 0 = 1 \pmod 2.
\]
Por tanto, $y \notin F^{-1}(V)$.

Esto ocurre para todo $N$, de modo que ningún vecindario derecho de $x_0$ está contenido en $F^{-1}(V)$.

\[
\boxed{F^{-1}(V) \text{ no es abierto } \Rightarrow F \text{ no continua.}}
\]

\subsection*{Interpretación}

En la topología de semirrectas derechas:
\begin{itemize}
  \item La noción de cercanía solo considera la parte ``futura'' de la configuración (índices grandes).
  \item La regla local, en cambio, depende de valores tanto a izquierda como a derecha.
  \item Por tanto, un pequeño cambio a la izquierda (invisible topológicamente) puede alterar el resultado global.
\end{itemize}

\[
\boxed{
\text{Topología producto (canónica): } F \text{ continua.}
\qquad
\text{Topología semirrectas derechas: } F \text{ no continua.}
}
\]

\bigskip

\section*{6. Comentario topológico final: compacidad, Cantor y estabilidad}

El espacio $X = A^{\mathbb{Z}^d}$, con $A$ finito y topología discreta, es un \textbf{producto infinito de espacios finitos discretos}.  
Por el \textit{teorema de Tychonoff}, $X$ es \textbf{compacto}.

Además, como es totalmente disconexo, sin puntos aislados y compacto, $X$ es \textbf{homeomorfo al espacio de Cantor}.  
Así, el espacio de configuraciones de un autómata celular tiene la misma estructura topológica que el famoso \textbf{conjunto de Cantor}.

\subsection*{Implicaciones dinámicas}

\begin{itemize}
  \item La continuidad de $F$ garantiza que la dinámica del autómata es \textbf{estable} respecto a perturbaciones locales.
  \item La compacidad de $X$ implica que toda secuencia de configuraciones tiene una subsecuencia convergente (en el sentido topológico).
  \item La estructura tipo Cantor permite aplicar herramientas de la \textbf{dinámica simbólica}, como medidas invariantes, entropía topológica y conjuntos invariantes.
\end{itemize}

En otras palabras, los autómatas celulares pueden verse como \textbf{transformaciones continuas en espacios de Cantor compactos}, lo que los sitúa dentro de la teoría general de los \textbf{sistemas dinámicos topológicos discretos}.

\[
\boxed{
(X,F) \text{ es un sistema dinámico topológico discreto y compacto.}
}
\]

\end{document}