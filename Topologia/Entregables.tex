\usepackage{asmath}
\title{Entregables Topología}
\author{Juan Rodríguez}
\date{}
\begin{document}
\maketitle
\section*{Tema 1: Distancias}

\subsection*{Ejercicio 1}
Determinar cuáles de las siguientes funciones son distancias en $\mathbb{R}$.

\begin{enumerate}[label=\alph*)]
\item $d(x,y) = |x - y|$. \\
Es una distancia: cumple todos los axiomas de una métrica (distancia usual).

\item $d(x,y) = |x^2 - y^2|$. \\
No es una distancia, ya que $d(1,-1)=0$ y $1\neq -1$. Falla el axioma de separación.

\item $d(x,y) = |x - 2y|$. \\
No es una distancia: no es simétrica ($d(x,y)\neq d(y,x)$ en general).

\item $d(x,y) = (x - y)^2$. \\
No es una distancia: falla la desigualdad triangular. Por ejemplo, $x=0,\, y=2,\, z=1$ da $4\nleq 1+1$.

\item $d(x,y) = \sin^2(x - y)$. \\
No es una distancia: $d(x,y)=0$ cuando $x - y = k\pi$, no solo cuando $x=y$.

\item $d(x,y) = \arctan|x - y|$. \\
Sí es una distancia. La función $\arctan$ es creciente, cóncava y subaditiva en $[0,\infty)$, por lo que se conserva la desigualdad triangular.
\end{enumerate}

\subsection*{Ejercicio 2}
Sea $d(A,B) = |A\cup B| - |A\cap B|$ definida sobre subconjuntos finitos de un conjunto universo.

\medskip
Se cumple que $|A\cup B| - |A\cap B| = |A\triangle B|$ (diferencia simétrica), que sí define una distancia: 
\[
d(A,B) \ge 0, \quad d(A,B)=0 \iff A=B, \quad d(A,B)=d(B,A),
\]
y además
\[
A\triangle C \subseteq (A\triangle B)\cup (B\triangle C) \implies |A\triangle C|\le |A\triangle B|+|B\triangle C|.
\]
Por tanto, $d$ es una métrica.

\subsection*{Ejercicio 3}
Analizar cuáles de las siguientes funciones son distancias en $\mathbb{R}^2$.

\begin{enumerate}[label=\alph*)]
\item $d((x_1,y_1),(x_2,y_2)) = \sqrt{(x_1-x_2)^2 + (y_1-y_2)^2}$. \\
Sí, es la distancia euclidiana.

\item $d((x_1,y_1),(x_2,y_2)) = |x_1 - x_2|\cdot |y_1 - y_2|$. \\
No, vale 0 si comparten $x$ o $y$.

\item $d((x_1,y_1),(x_2,y_2)) = |x_1 - x_2| + |y_1 - y_2|$. \\
Sí, es la métrica Manhattan ($\ell^1$).

\item $d((x_1,y_1),(x_2,y_2)) = \min(|x_1 - x_2|, |y_1 - y_2|)$. \\
No, puede valer 0 para puntos distintos.

\item $d((x_1,y_1),(x_2,y_2)) = \max(|x_1 - x_2|, |y_1 - y_2|)$. \\
Sí, es la métrica de Chebyshev ($\ell^\infty$).
\end{enumerate}

\subsection*{Ejercicio 4}
Sea $d(x,y) = \min(|x-y|, 1 - |x-y|)$ en $[0,1)$. 

\medskip
Esta es la distancia circular (geodésica en $S^1$). Cumple todos los axiomas de una métrica, por lo que $d$ es una distancia.

\subsection*{Ejercicio 5}
Determinar la forma de una distancia en un conjunto finito que solo tome valores 0 o 1.

\medskip
La única posibilidad es la \textbf{métrica discreta}:
\[
d(x,y) = 
\begin{cases}
0, & \text{si } x=y,\\
1, & \text{si } x\neq y.
\end{cases}
\]

\subsection*{Ejercicio 6}
Comparar las métricas Manhattan y Euclidiana en $\mathbb{R}^2$.

\medskip
Para todo $v=(x,y)$ se cumple:
\[
\|v\|_2 \le \|v\|_1 \le \sqrt{2}\,\|v\|_2.
\]
Por tanto, ambas son \textbf{equivalentes}, ya que inducen la misma topología.

\subsection*{Ejercicio 7}
Sea $A,B,C\in\mathbb{R}^2$. Analizar la existencia de un punto $O$ equidistante de los tres puntos.

\begin{enumerate}[label=\alph*)]
\item En la métrica taxicab continua ($\ell^1$), siempre existe al menos un punto $O$ equidistante (no necesariamente único).
\item En la métrica taxicab discreta (sobre $\mathbb{Z}^2$), existe tal punto $O\in\mathbb{Z}^2$ si y solo si
\[
a_1+a_2 \equiv b_1+b_2 \equiv c_1+c_2 \pmod 2.
\]
\end{enumerate}

\subsection*{Ejercicio 8}
Calcular el valor análogo a $\pi$ en la métrica taxicab.

\medskip
En $\ell^1$, el círculo de radio $r$ es un rombo de perímetro $8r$ y diámetro $2r$, luego
\[
\pi_{\text{taxi}} = \frac{\text{perímetro}}{\text{diámetro}} = \frac{8r}{2r} = 4.
\]

\subsection*{Ejercicio 9}
Sea $f(x)=\sin(2x)$ y $g(x)=\cos(x)$ en $[0,\pi]$. Calcular las distancias siguientes:

\begin{enumerate}[label=\alph*)]
\item \textbf{Métrica integral:}
\[
d_1(f,g) = \int_0^\pi |\sin 2x - \cos x|\,dx 
= \int_0^\pi |\cos x\,(2\sin x - 1)|\,dx = 1.
\]

\item \textbf{Métrica del supremo:}
\[
d_\infty(f,g) = \sup_{x\in[0,\pi]} |\sin 2x - \cos x|
= \sup |\cos x\,(2\sin x - 1)| = 1.
\]
\end{enumerate}

\end{document}