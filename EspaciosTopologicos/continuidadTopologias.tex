\documentclass[fleqn]{article}
\usepackage{amsmath}
\usepackage{amsfonts}

\title{Continuidad en Topologías}
\author{Juan Felipe Rodríguez Córdoba}
\date{}

\begin{document}
	\maketitle
    \section*{Definición de Continuidad}
    Sea \( f: X \to Y \) una función entre dos espacios topológicos \( (X, \tau_X) \) y \( (Y, \tau_Y) \).
    
    Decimos que \( f \) es continua si para cada conjunto abierto \( V \in \tau_Y \), su preimagen \( f^{-1}(V) \) es un conjunto abierto en \( X \), es decir, \( f^{-1}(V) \in \tau_X \).

    \section*{Definición Función Inversa}
    La función inversa de \( f \), denotada como \( f^{-1} \), es una función que asigna a cada conjunto \( V \subseteq Y \) su preimagen en \( X \):
    \[ f^{-1}(V) = \{ x \in X \mid f(x) \in V \}. \]

    \section*{Topología de Semirrectas Derechas}
    La topología de semirrectas derechas en \( \mathbb{R} \) está definida por la base de conjuntos abiertos de la forma \( (a, \infty) \).

    \section*{Ejemplo en Topología de Semirrectas Derechas}
    Consideremos los espacios topológicos \( (X, \tau_X) \) y \( (Y, \tau_Y) \) donde:
    \begin{itemize}
        \item \( X = \mathbb{R} \) con la topología de semirrectas derechas.
        \item \( Y = \mathbb{R} \) con la topología de semirrectas derechas.
    \end{itemize}
    Definimos el conjunto abierto en \( Y \):
    \[ A = (a, \infty) \]
    \begin{itemize}
        \item \(f(x) = 2x\) \\
        \(f^{-1}(A) = (\frac{a}{2}, \infty)\) es abierto en \( X \). Por lo tanto, \( f \) es continua.
        \item \(f(x) = -x\) \\
        \(f^{-1}(A) = (-\infty, -a)\) no es abierto en \( X \). Por lo tanto, \( f \) no es continua.
        \item \(f(x) = x^2\) \\
        \(f^{-1}(A) = (-\infty, -\sqrt{a}) \cup (\sqrt{a}, \infty)\) no es abierto en \( X \). Por lo tanto, \( f \) no es continua.
        \item \(f(x) = x^3\) \\
        \(f^{-1}(A) = (\sqrt[3]{a}, \infty)\) es abierto en \( X \). Por lo tanto, \( f \) es continua.
    \end{itemize}

    \section*{Topología de Sorgenfrey}
    La topología de Sorgenfrey en \( \mathbb{R} \) está definida por la base de conjuntos abiertos de la forma \( [a, b) \) donde \( a < b \).
    
    \section*{Ejemplo en Topología de Sorgenfrey}
    Consideremos los espacios topológicos \( (X, \tau_X) \) y \( (Y, \tau_Y) \) donde:
    \begin{itemize}
        \item \( X = \mathbb{R} \) con la topología de Sorgenfrey.
        \item \( Y = \mathbb{R} \) con la topología de Sorgenfrey.
    \end{itemize}
    Definimos el conjunto abierto en \( Y \):
    \[ A = [a, b) \]
    \begin{itemize}
        \item \(f(x) = 2x\) \\
        \(f^{-1}(A) = [\frac{a}{2}, \frac{b}{2})\) es abierto en \( X \). Por lo tanto, \( f \) es continua.
        \item \(f(x) = -x\) \\
        \(f^{-1}(A) = (-b, -a]\) no es abierto en \( X \). Por lo tanto, \( f \) no es continua.
        \item \(f(x) = x^2\) \\
        \(f^{-1}(A) = (-\sqrt{b}, -\sqrt{a}] \cup [\sqrt{a}, \sqrt{b})\) no es abierto en \( X \). Por lo tanto, \( f \) no es continua.
        \item \(f(x) = x^3\) \\
        \(f^{-1}(A) = [\sqrt[3]{a}, \sqrt[3]{b})\) es abierto en \( X \). Por lo tanto, \( f \) es continua.
    \end{itemize}

    \section*{Topología Cofinita}
    La topología cofinita en un conjunto \( X \) está definida por la colección de conjuntos abiertos que son \( X \) y todos los subconjuntos cuyo complemento en \( X \) es finito.

    \section*{Ejemplo en Topología Cofinita}
    Todas las funciones \( f: X \to Y \) entre espacios topológicos con la topología cofinita son continuas excepto las periodicas. Esto se debe a que la preimagen de cualquier conjunto abierto en \( Y \) (que es \( Y \) o tiene complemento finito) tendrá un complemento finito en \( X \), asegurando que la preimagen sea abierta en \( X \).
\end{document}